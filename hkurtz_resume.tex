%-------------------------
% Resume in Latex
% Author : Harrison Kurtz
% License : MIT
%------------------------

\documentclass[letterpaper,11pt]{article}

\usepackage{latexsym}
\usepackage[empty]{fullpage}
\usepackage{titlesec}
\usepackage{marvosym}
\usepackage[usenames,dvipsnames]{color}
\usepackage{verbatim}
\usepackage{enumitem}
\usepackage[hidelinks]{hyperref}
\usepackage{fancyhdr}
\usepackage[english]{babel}
\usepackage{tabularx}
\input{glyphtounicode}

\pagestyle{fancy}
\fancyhf{} % clear all header and footer fields
\fancyfoot{}
\renewcommand{\headrulewidth}{0pt}
\renewcommand{\footrulewidth}{0pt}

% Adjust margins
\addtolength{\oddsidemargin}{-0.5in}
\addtolength{\evensidemargin}{-0.5in}
\addtolength{\textwidth}{1in}
\addtolength{\topmargin}{-.5in}
\addtolength{\textheight}{1.0in}

\urlstyle{same}

\raggedbottom
\raggedright
\setlength{\tabcolsep}{0in}

% Sections formatting
\titleformat{\section}{
  \vspace{-4pt}\scshape\raggedright\large
}{}{0em}{}[\color{black}\titlerule \vspace{-5pt}]

% Ensure that generate pdf is machine readable/ATS parsable
\pdfgentounicode=1

%-------------------------
% Custom commands
\newcommand{\resumeItem}[2]{
  \item\small{
    \textbf{#1}{: #2 \vspace{-2pt}}
  }
}

% Just in case someone needs a heading that does not need to be in a list
\newcommand{\resumeHeading}[4]{
    \begin{tabular*}{0.99\textwidth}[t]{l@{\extracolsep{\fill}}r}
      \textbf{#1} & #2 \\
      \textit{\small#3} & \textit{\small #4} \\
    \end{tabular*}\vspace{-5pt}
}

\newcommand{\resumeSubheading}[4]{
  \vspace{-1pt}\item
    \begin{tabular*}{0.97\textwidth}[t]{l@{\extracolsep{\fill}}r}
      \textbf{#1} & #2 \\
      \textit{\small#3} & \textit{\small #4} \\
    \end{tabular*}\vspace{-5pt}
}

\newcommand{\resumeSubSubheading}[2]{
    \begin{tabular*}{0.97\textwidth}{l@{\extracolsep{\fill}}r}
      \textit{\small#1} & \textit{\small #2} \\
    \end{tabular*}\vspace{-5pt}
}

\newcommand{\resumeSubheadingContinue}[2]{
  \vspace{-1pt}
    \begin{tabular*}{0.97\textwidth}[t]{l@{\extracolsep{\fill}}r}
      \textit{\small#1} & \textit{\small #2} \\
    \end{tabular*}\vspace{-5pt}
}

\newcommand{\resumeSubItem}[2]{\resumeItem{#1}{#2}\vspace{-4pt}}

\renewcommand{\labelitemii}{$\circ$}

\newcommand{\resumeSubHeadingListStart}{\begin{itemize}[leftmargin=*]}
\newcommand{\resumeSubHeadingListEnd}{\end{itemize}}
\newcommand{\resumeItemListStart}{\begin{itemize}}
\newcommand{\resumeItemListEnd}{\end{itemize}\vspace{-5pt}}

%-------------------------------------------
%%%%%%  CV STARTS HERE  %%%%%%%%%%%%%%%%%%%%%%%%%%%%


\begin{document}

%----------HEADING-----------------
\begin{tabular*}{\textwidth}{l@{\extracolsep{\fill}}r}
  \textbf{\href{https://hkurtz.com/}{\Large Harrison Kurtz}} & Email : \href{mailto:jobs@hkurtz.com}{jobs@hkurtz.com}\\
  & Mobile : +1-214-206-6313 \\
\end{tabular*}

%-----------EXPERIENCE-----------------
\section{Experience}
  \resumeSubHeadingListStart

    \resumeSubheading
      {IBM}{Austin, TX}
      {Development Lead, Maximo Application Suite (MAS)}{Jan 2020 - Present}        
      \resumeItemListStart
        \resumeItem{Suite Licensing Service}
          {Designed and developed Suite Licensing Service (SLS), a token based licensing system that powers licensing for MAS and other IBM application suites. Lead a small team in API implementation, operator development, and testing.}
        \resumeItem{Suite Licensing Integration}
          {Integrated licensing into MAS's catalog inventory management, user registry management, and single sign-on flows.}
      \resumeItemListEnd

      \resumeSubheadingContinue
        {Software Engineer, Watson IoT Platform (WIoTP)}{Jan 2015 - Dec 2019}
        \resumeItemListStart
          \resumeItem{Blue/Green Deployment}
            {Added blue/green deployment capabilities to the platform's deployment pipeline to minimize risks and disruptions caused by messaging infrastructure deployments.}
          \resumeItem{Client State}
            {Developed APIs in Java that provided real-time connection status for MQTT clients. Created, ran, and analyzed performance tests for client state feature.}
          \resumeItem{Messaging Infrastructure Kubernetes Migration}
            {Created kubernetes deployment for messaging infrastructure. Migrated MQTT servers, MQTT proxies, and loadbalancers in production from a VM environment to a kubernetes environment with minimal downtime.}
          \resumeItem{SRE}
            {Joined the SRE chapter. Responsible for incident management and customer support for production environments during on-call rotation. Developed, operated, and maintained platform build and deployment pipelines.}
          \resumeItem{Continuous Loadtest}
            {Designed, developed, and maintained tests that validate various platform features' health and performance. Created dashboards and rules based on acceptable performance standards to monitor the platform's health and performance over time.}
          \resumeItem{Componetization}
            {Assisted in breaking down the platform's monolithic build system, deployment system, and database design. Researched and provided guidelines for microservice architecture. Transitioned platform's monolithic architecture to microservice architecture in piecemeal fashion.}
          \resumeItem{Mongo}
            {Became team expert in Mongo. Wrote deployment automation, developed monitoring and metrics agents, rewrote database layer for existing platform services to use Mongo, and migrated platform data from previous database solution to Mongo.}
          \resumeItem{Device Management}
            {Assisted in creating the WIoTP device management protocol, a protocol built on the MQTT messaging protocol. Developed a device management server in Java that implemented the protocol. Added device management functionality to Python, Java, and Node.js device clients.}
        \resumeItemListEnd
  
    \resumeSubheading
      {Texas Instruments}{Dallas, TX}
      {Product Engineer Intern}{Summer 2013}
      \resumeItemListStart
        \resumeItem{Adaptive Body Bias}
          {Wrote and ran tests that applied Adaptive Body Bias on TI's DM385 DSP to contrast the improvments in performance with the current leakage of the chip.}
      \resumeItemListEnd

  \resumeSubHeadingListEnd

%-----------EDUCATION-----------------
\section{Education}
  \resumeSubHeadingListStart
    \resumeSubheading
      {Texas A\&M University}{College Station, TX}
      {Bachelor of Science in Computer Engineering}{Aug 2010 - Dec 2014}
  \resumeSubHeadingListEnd

%-----------Awards and Publications-----------------
\section{Awards and Publications}
  \resumeSubHeadingListStart
    \resumeSubItem{US Patent US20180309614A1}
      {Devices Demise Actions and Notification}
    \resumeSubItem{US Patent US20180302412A1}
      {Logical Zones for IoT Devices}
    \resumeSubItem{Generation Google Scholarship}
      {Awarded through Google and American Indian Science and Engineering Society (AISES).}
  \resumeSubHeadingListEnd

%--------PROGRAMMING SKILLS------------
\section{Programming Skills}
  \resumeSubHeadingListStart
    \item{
      \textbf{Languages}{: Java, Python, Javascript}
    }
    \item{
      \textbf{Frameworks}{: Operator SDK, Jakarta EE, JUnit, JMeter, Flask, pytest}
    }
    \item{
      \textbf{Technologies}{: Kubernetes, MQTT, Ansible, Mongo, Cassandra, Kafka, IBM Cloud}
    }
  \resumeSubHeadingListEnd


%-------------------------------------------
\end{document}
